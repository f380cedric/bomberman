\newpage
\section{Problématique et approche}
Comme cité dans l'abstract, la problématique posée dans le cadre de ce projet est de réaliser un prototype du jeu \textit{Bomberman}, en utilisant les concepts de l'orientée objet que nous avons rencontrés au cours oral et lors des travaux pratiques. Le jeu doit être conçu dans le but d'y jouer en multijoueur uniquement, les concepts d'intelligence artificielle n'ayant pas été abordés. Tout en fournissant les fonctionnalités de base du \textit{Bomberman}, les étudiants sont libres d'implémenter les fonctionnalités qu'ils veulent. \\
Bien que chacun des membres de notre groupe avait une certaine 
connaissance de l'informatique (notamment grâce au cours de 
première), aucun d'entre nous n'avait auparavant pratiqué la  programmation orientée objet.


	\subsection{Approche}
	N'ayant aucune connaissance dans ce paradigme qu'est la 
	programmation orientée objet, notre première approche
	a été d'acquérir les bases théoriques nécessaires.\\
	Après cela, nous avions une bonne compréhension
	théorique des concepts, mais aucune expérience au niveau 
	pratique. Afin de ne pas nous lancer dans l'inconnu et pour 
	exploiter au maximum nos connaissances théoriques, nous avons 
	commencé, avant même d'écrire une seule ligne de code, par 
	réaliser le diagramme des classes que nous avons présenté 
	dans notre prérapport.\\
	Ce travail préliminaire nous a été d'une grande utilité, car 
	nous avons pu développer le jeu de façon rapide et structurée.
	À l'exception de quelques bonus rajoutés et l'une ou l'autre
	légère modification, notre diagramme de classe final est 
	semblable à 85\% à l'initial.\\
	
	Le fait que la structure ait été pensée avant toute chose nous
	a permis de ne rencontrer qu'un faible nombre de bugs lors du
	développement, mais aussi de pouvoir coder assez rapidement les bases
	du jeu. Cependant, par la suite, nous avons été confrontés à un problème
	autre que
	le manque de connaissance : notre jeu était extrêmement gourmand 
	en ressources. Le problème était lié au 
	rafraîchissement de notre interface graphique qui se faisait à l'aide d'un
	\textit{Thread} à petit intervalle de temps. Pour résoudre ce
	problème, nous avons procédé à trois changements
	majeurs :
	\begin{enumerate}
	\item La grille complète n'est chargée qu'une seule fois, lors 
	de l'initialisation du jeu. 
	\item L'affichage n'est plus actualisé de façon continue
	\item Plutôt que de rafraîchir l'intégralité de la fenêtre, 
	seules les modifications du plateau de jeu sont \textit{repaint} sur les anciens éléments par le biais d'une case blanche pour cacher ce dernier, si nécessaire.
	\end{enumerate}
	
	Ces changements nous ont permis d'améliorer considérablement notre
	consommation des ressources de l'ordinateur et constituent l'un des
	points forts de ce projet.\\
	Les deux points présentés ici étaient de loin les plus grosses
	problématiques rencontrées. Bien sûr, nous avons connu plusieurs
	problèmes isolés, mais ceux-ci étaient plus des "petits bugs" 
	qu'une véritable problématique. 
	
	\subsection{Répartition du travail}
	Au niveau de la répartition du travail, nous n'avons pas	séquencé le projet en parties distinctes, mais nous avons 
	travaillé parallèlement sur chacune des parties du projet (
	la collaboration nous étant grandement facilité par l'
	utilisation de \textit{git}	). Par exemple, au niveau de la 
	gestion de la bombe, nous avons procédé de la façon suivante :
	\begin{description}
	\item[Nico] s'est chargé du dépôt de la bombe et de l'auto-
	destruction de cette dernière.
	\item[Enes] s'est chargé de la destruction des blocs causés 
	par l’explosion de la bombe ainsi que la génération aléatoire
	des bonus.
	\item[Cédric] s'est chargé de l'auto déclenchement d'une bombe
	lorsque celle-ci se fait elle-même exploser par une autre 
	bombe.
	\end{description}
	
	Cet exemple de la bombe a été appliqué à la quasi-totalité du code. 
	Cette approche nous a donc permis à chacun d'avoir une connaissance 
	de l'intégralité du code et d'ainsi savoir nous aider de 
	façon rapide lors de l'apparition d'un bug.

